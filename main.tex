\documentclass [xcolor=svgnames, t] {beamer} 
\usepackage[utf8]{inputenc}
\usepackage{booktabs, comment} 
\usepackage[absolute, overlay]{textpos} 
\usepackage{pgfpages}
\usepackage[font=footnotesize]{caption}
\useoutertheme{infolines} 

\definecolor{gold}{RGB}{254, 206, 0}

%\setbeamercolor{title in head/foot}{bg=gold, fg=black}
\setbeamercolor{author in head/foot}{bg=myuniversity}
\setbeamertemplate{page number in head/foot}{}
\usepackage{csquotes}
\setbeamertemplate{navigation symbols}{}

\usepackage{amsmath}
\usepackage[makeroom]{cancel}


\usepackage{textpos}

\usepackage{tikz}
\usepackage{multirow}

\usetheme{Dresden}
\setbeamertemplate{footline}[]
\definecolor{myuniversity}{RGB}{0, 85, 150}
\usecolortheme[named=myuniversity]{structure}
\usepackage{tikz}



\title[]{A New Formulation of Minimum Risk Fixed-Width Confidence
Interval (MRFWCI) for a Normal Mean}
\titlegraphic{\includegraphics[height=2 cm]{UConnlogo.png}}
\institute[]{Department of Statistics \\ University of Connecticut}

\author[]{ Swathi Venkatesan }

\date{}



\addtobeamertemplate{navigation symbols}{}{%
    \usebeamerfont{footline}%
    \usebeamercolor[fg]{footline}%
    \hspace{1em}%
    \insertframenumber/\inserttotalframenumber
}

\begin{document}
\begin{frame}
\maketitle
\end{frame}




%%%%%%%%%%%%%%%%%%%%%%%%%%
%\begin{frame}
%\frametitle{Table of Contents}
%\tableofcontents
%\end{frame}
%%%%%%%%%%%%%%%%%%%%%%%%%%
\section{General Theory of MRFWCI}
%\section{Abstract}
%\begin{frame}{Motivation}
\vspace{10mm}
 %The fixed-width confidence interval (FWCI) estimation problems for a normal mean when the variance is unknown have moved along under a zero-one loss without taking into account sampling cost. While, minimum risk point estimation (MRPE) problems have grown largely under squared error loss (SEL) plus sampling cost. Here, a new formulation combining both MRPE and FWCI methodologies is introduced with desired asymptotic second-order characteristics under a unified structure to develop a minimum risk fixed-width confidence interval (MRFWCI) strategy.
%\end{frame}

%%%%%%%%%%%%%%%%%%%%%%%%%%
%\section{Loss and Risk Function}
\begin{frame}{Construction of Loss and Risk Functions}
    Suppose $X_1,X_2, \dots ,X_n \dots$ are \textit{independent and identically distributed (i.i.d)} observations from a common $N(\mu,\sigma^2)$ distribution where $\mu \in \mathbb{R}$, $\sigma \in \mathbb{R}^{+}$, $\theta=(\mu,\sigma)$ and both $\mu,\sigma$ are unknown. \\
    \vspace{2mm}
    Having pre-assigned numbers, $d(>0)$ and $0<\alpha<1$, the FWCI for $\mu$ is:
        $$\textbf{FWCI: } J_n \equiv \left[ \bar{X}_n \pm d\right],$$
        \begin{equation} \label{opt_n} 
        \begin{split}
        & \text{ with confidence coefficient }P_{\theta}\{\mu \in J_n\} = 2\Phi(n^{1/2}d/\sigma)-1  \ge 1-\alpha \\
        & \text{ when } n \text{ is the smallest integer } \ge z^2_{\alpha/2}\sigma^2 d^{-2} \equiv n^*.
        \end{split}
        \end{equation}

\end{frame}
%%%%%%%%%%%%%%%%%%%%%%%%%%
\begin{frame}{}
    \vspace{7mm}
    Generically, we may initiate the following idea of a loss function in practice:
        \begin{equation} \label{loss_fn}
            L_n(\theta)  = EstErr_n(\theta)+c_n(\theta),
        \end{equation}
        where $EstErr_n(\theta)$ is the estimation error involving $I \left[ {\tau(\theta) \not\in FWCI;n} \right]$ plus the $c_n(\theta) \equiv \text{sampling cost}_n$\\
        \vspace{2mm}
    We combine (1) and (2) to express:
        \begin{equation}
        p_n(\theta ;J_n) \equiv E_{\theta} \{ {I[ \mu \not\in J_n;n]}\} =2\{ 1-\Phi(n^{1/2}d/\sigma) \}.
        \end{equation}
        We incorporate the following expression of the cost function:
        \begin{equation} \label{cost_fn}
        c_n(\theta) = 2(d^\rho \sigma)^{-1} \phi(z_{\alpha/2})n^{1/2}.
        \end{equation}
        
\end{frame}
%%%%%%%%%%%%%%%%%%%%%%%%%%
\begin{frame}{}
\vspace{16mm}
Then, in the spirits of (1) and (3)-(4), we construct the loss function (for fixed $0<\rho \le 1$) as:
        \begin{equation}
            \textbf{Loss: }L_n(\mu,J_n) \equiv d^{-\rho-1}I[\mu \not\in J_n;n]+c_n(\theta).
        \end{equation}
        The corresponding Risk is given by:
    \begin{equation*}
    \begin{split}
        \textbf{Risk: } R_n(\mu,J_n)  \equiv E_{\theta}[L_n(\mu,J_n)] \\ &= 2d^{-\rho-1}[1-\Phi(n^{1/2}d/\sigma)]+2(d^{\rho}\sigma)^{-1}\phi(z_{\alpha/2})n^{1/2} .
    \end{split}
    \end{equation*}
    
\end{frame}
%%%%%%%%%%%%%%%%%%%%%%%%%%

%%%%%%%%%%%%%%%%%%%%%%%%%%
%\section{General Theory-MRFWCI}
\begin{frame}{General setup of MRFWCI-Normal mean Estimation}

We propose the MRFWCI for $\mu$ based on the recorded observation $\{N,X_1,X_2,\dots ,X_N\}$ as:
    \begin{equation*}
    \textbf{MRFWCI: }J_N=[\bar{X}_N \pm d] \text{ where }\bar{X}_N=N^{-1} \sum_{i=1}^N{X_i}.    
    \end{equation*}
    
    The risk function $R_N(\mu,J_N)$ with the terminal strategy $(N,J_N)$ for $\mu$ under (5) is:
    \begin{equation*}
        R_N(\mu,J_N) = 2\{ d^{-\rho-1} E_{\theta}[1-\Phi(N^{1/2}d/\sigma)]+ (d^{\rho}\sigma)^{-1}\phi(z_{\alpha/2})E_{\theta}(N^{1/2})\}.
    \end{equation*}
    
    %The optimal fixed sample size (had $\sigma$ been known) corresponding to the minimum risk associated with the MRFWCI formulation under the new loss function remains exactly the same as that shown earlier in (1).
    
\end{frame}

%%%%%%%%%%%%%%%%%%%%%%%%%%

%%%%%%%%%%%%%%%%%%%%%%%%%%
%\section{Properties}
\begin{frame}{}
\vspace{15mm}
 Without specifying a formal sampling strategy , we assume a number of key associated properties as $d \xrightarrow{} 0$ :
   
    \begin{itemize}
        \item[\textbf{A1}] $ E_{\theta}(N)=n^* +a_1 +o(1)$;
         \item[\textbf{A2}] $H \equiv \frac{N-n^*}{n^{*1/2}} \xrightarrow{\L} N(0,a_2), a_2>0$ ;
          \item[\textbf{A3}] $P_{\theta}\{ N \le \epsilon^*\}=O(n^{*-a_3}),\epsilon \in (0,1), a_3>0$;
           \item[\textbf{A4}] $|H|^{a_4} \text{ is uniformly integrable },a_4 >0$.
    \end{itemize}
    
    where the expressions of the real numbers $a_i's$ would not involve $d$.
\end{frame}
%%%%%%%%%%%%%%%%%%%%%%%%%%
\begin{frame}{Asymptotic first-order properties}
\vspace{7mm}
For the general MRFWCI estimation methodology carried out under the loss function (5), for every fixed $\alpha, \rho \text{ and } \theta$, with $a_i$ defined via assumption $A_i$, we have the following asymptotic first-order conclusions as $d \xrightarrow{} 0$ :

\begin{itemize}
\item[(i)] $n^{*-1}N \xrightarrow[]{P_{\theta}}1$;
\item[(ii)] $E_{\theta}[(n^{*-1}N)^{\kappa}] \xrightarrow[]{} 1$ [\textit{asymptotic first-order efficiency property}];
\item[(iii)] $P_{\theta} \{ \mu \in J_N\} \xrightarrow[]{} (1-\alpha)$ [\textit{asymptotic consistency property}];
\item[(iv)] $\text{RiskEff}_d =\frac{R_N(\mu,J_N)}{R_{n^*}(\mu,J_{n^*})} \xrightarrow[]{} 1$ [\textit{asymptotic risk efficiency property}].
\end{itemize}
 
\end{frame}
%%%%%%%%%%%%%%%%%%%%%%%%%%
\begin{frame}{Asymptotic second-order properties}
\vspace{5mm}
For the general MRFWCI estimation methodology carried out under (5), for every fixed $\alpha, \rho \text{ and } \theta$, with $a_i$ defined via assumption $A_i$, we have the following asymptotic second-order conclusions when $a_3>\frac{5}{2}$ and $a_4=2$ as $d \xrightarrow{} 0$ :
\begin{enumerate}
\item[(i)] $E_{\theta}[(n^{*-1}N)^{1/2}]=1+a_5n^{*-1}+o(n^{*-1})$ where $a_5=\frac{1}{2}(a_1-\frac{1}{2}a_2)$;
\item[(ii)] $P_{\theta} \{ \mu \in J_N\} = (1-\alpha)+2a_6n^{*-1}+o(n^{*-1)}$\\
where $a_6=\frac{1}{2}\{ a_1-(a_2/4)(1-z^2_{\alpha/2}) \}z_{\alpha/2}\phi(z_{\alpha/2})$;
\item[(iii)] $\text{RiskEff}_d =\omega=\frac{R_N(\mu,J_N)}{R_{n^*}(\mu,J_{n^*})}= 1+ a_7 n^{*-1}+o(n^{*-1)}$ \\ where $a_7 = 2\{\alpha + 2z_{\alpha/2}\phi(z_{\alpha/2})\}^{-1} \{z_{\alpha/2}\phi(z_{\alpha/2})a_5-a_6\}$.
\end{enumerate}
 
\end{frame}
%%%%%%%%%%%%%%%%%%%%%%%%%%

%%%%%%%%%%%%%%%%%%%%%%%%%%
\section{Sampling Strategies}
%\section{Sampling Strategies}
\begin{frame}{Illustrations of Multistage Sampling Strategies}
    \begin{itemize}
    \item \textbf{Purely Sequential Sampling strategy}
\\ We begin with pilot observations $X_1,\dots ,X_m; m\ge 2$. Then, we record one additional observation at-a-time according to the following stopping
time:
\begin{equation*}
    N \equiv N_d=inf\{ n \ge m; n \ge z^2_{\alpha/2}S_n^2/d^2 \}  \textrm{ and }  J_N=[\bar{X}_N \pm d].
\end{equation*}
\item \textbf{Two-stage Sampling strategy}
    \\ We assume that there exists known $\sigma_L(>0)$ such that $\sigma>\sigma_L$.The pilot sample size is defined as:
    $$m\ equiv m_d=max \{ m_0(\ge 2),<z^2_{\alpha/2}\sigma_L^2/d^2>+1\}.$$
    We begin with pilot observations $X_1,\dots ,X_m, m \ge 2$ and find the final sample size as:
    \begin{equation*}
    N \equiv N_d=max\{ m,< t^2_{m-1,\alpha/2}S_m^2/d^2>+1\} \text{ and } J_N=[\bar{X}_N \pm d].
\end{equation*}
    \end{itemize}
\end{frame}
%%%%%%%%%%%%%%%%%%%%%%%%%%

\begin{frame}{}
    \vspace{20mm}
    The table below summarizes some properties of the sampling strategies.
\begin{center}
\begin{tabular}{c c c} 
 \hline
 Sampling Strategy & $a_1$ & $a_2$ \\
 \hline
 Purely Sequential & $-1.1828$ & $2$  \\ 
 Two-stage & $\frac{1}{2}[(z^2_{\alpha/2}+1)\frac{\sigma^2}{\sigma^2_L}+1]$ & $2\frac{\sigma^2}{\sigma^2_L}$ \\
 \hline
\end{tabular}
\end{center}
\end{frame}
%%%%%%%%%%%%%%%%%%%%%%%%%%
\section{Data Analysis and Simulations}
%\section{Data Analysis and Simulations}
\begin{frame}{\textit{Airquality} Data Analysis}

Now we will illustrate applications of the MRFWCI problem for the average in the context of \textit{airquality dataset} describing daily air quality measurements in New York. The variable of interest $(X)$ was \textit{wind speed},which was normally distributed. The mean and standard deviation from the full dataset were: $\mu = 9.957$ and $\sigma = 3.523$. The table below summarises the performance of the sampling strategies (fixing $\alpha=0.05$ and $\rho=1.0$). The notations are defined below:
\begin{itemize}
\item $n$: terminal sample size;
\item $\hat{\mu}_n=n^{-1}\sum_{j=1}^n{x_j}$; terminal sample mean;
\item $J_n=[\hat{\mu}_n \pm d]$; terminal $95\%$ MRFWCI for $\mu$.
\end{itemize}
    
\end{frame}

\begin{frame}{}
    \vspace{8mm}
    \begin{center}
\begin{tabular}{c c c c c} 
 \hline
 Sampling Strategy & $d$ & $n$ & $\hat{\mu}_n$ & $J_n$  \\  
 \hline
{Purely Sequential}  & $0.7$ & $102$ & $10.034$ & $[9.334,10.734]$  \\ 
 $m=15$ & $0.8$ & $83$ & $10.427$ & $[9.626,11.226]$ \\
  & $0.9$ & $65$ & $10.795$ & $[9.895,11.695]$\\
 \hline
{Two-stage }  & $0.7$ & $105$ &$10.072$ & $[9.375,10.775]$  \\ 
 $\sigma_L=2,m_0=5$ & $0.8$ & $79$ & $10.534$ & $[9.734,11.334]$ \\
  & $0.9$ & $69$ & $10.559$ & $[9.965,11.459]$\\
 \hline
{Two-stage }  & $0.7$ & $115$ & $10.146$ & $[9.446,10.846]$  \\ 
 $\sigma_L=3,m_0=5$ & $0.8$ & $90$ & $10.321$ & $[9.521,11.121]$ \\
  & $0.9$ & $56$ & $10.993$ & $[10.093,11.893]$\\
 \hline
\end{tabular}
\end{center}
\end{frame}
%%%%%%%%%%%%%%%%%%%%%%%%%%
\begin{frame}{Simulation Study}

Simulated performances of  MRFWCI strategies under $10,000$ replications for a $N(25,16)$ population with $\alpha=0.05$ and $\rho=1.0$ in (5).
 \begin{center}
\begin{tabular}{c c c c c} 
 \hline
 $n^*$ & $d$ & $\bar{n}$ & $\bar{p}$ & $\bar{\omega}$\\
 \hline
 \multicolumn{5}{c}{Purely Sequential $m=15$}\\
  \hline
 $200$ & $0.5543$ & $198.744$ & $0.9484$  & $1.004$\\ 
 $1000$ & $0.2479$ & $998.048$ & $0.9462$ & $1.001$ \\ 
  $5000$ & $0.1108$ & $4997.536$ & $0.9477$ & $1.001$ \\ 
  \hline
\multicolumn{5}{c}{Two-Stage $\sigma_L=3, m_0=10$}\\
 \hline
 $200$ & $0.5543$ & $200.851$ & $0.9459$ & $1.007$ \\ 
 $1000$ & $0.2479$ & $1001.101$ & $0.9459$  & $1.003$\\ 
  $5000$ & $0.1108$ & $4997.536$ & $0.9477$  & $1.001$\\ 
 \hline
\end{tabular}
\end{center}
    
\end{frame}
%%%%%%%%%%%%%%%%%%%%%%%%%%


%%%%%%%%%%%%%%%%%%%%%%%%%%
\section{Conclusion}
\begin{frame}{Conclusion}
\vspace{10mm}
When handling FWCI problems,the lack of a loss function creates a false impression that perhaps (i) observations may not cost at all and/or (ii) one's available budget may be unlimited. In the MRFWCI formulation,an FWCI problem has been casted in the light of an MRPE by balancing the estimation error with the cost of sampling.
\end{frame}
%%%%%%%%%%%%%%%%%%%%%%%%%%
%\section{Reference}

%\nocite{*}
\begin{frame}[allowframebreaks]{Reference}
%\bibliographystyle{unsrt}
%\bibliography{bibfile.bib}

\textbf{Nitis Mukhopadhyay and Swathi Venkatesan}.\\
\vspace{0.2cm}
A New Formulation of Minimum Risk Fixed-Width Confidence Interval
(MRFWCI) Estimation Problems for a Normal Mean with Illustrations
and Simulations: Applications to Airquality Data.\\
\textit{Sequential Analysis, 41, 2022}.

\end{frame}
%%%%%%%%%%%%%%%%%%%%%%%%%%


\end{document}